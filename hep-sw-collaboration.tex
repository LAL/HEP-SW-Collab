% !TEX TS-program = pdflatex
% !TEX encoding = UTF-8 Unicode

\documentclass[11pt]{article} % use larger type; default would be 10pt

\usepackage[utf8]{inputenc} % set input encoding (not needed with XeLaTeX)

%%% Examples of Article customizations
% These packages are optional, depending whether you want the features they provide.
% See the LaTeX Companion or other references for full information.

%%% PAGE DIMENSIONS
\usepackage{geometry} % to change the page dimensions
\geometry{a4paper} % or letterpaper (US) or a5paper or....
% \geometry{margin=2in} % for example, change the margins to 2 inches all round
% \geometry{landscape} % set up the page for landscape
%   read geometry.pdf for detailed page layout information

\usepackage{graphicx} % support the \includegraphics command and options

% \usepackage[parfill]{parskip} % Activate to begin paragraphs with an empty line rather than an indent

%%% PACKAGES
%\usepackage{booktabs} % for much better looking tables
\usepackage{array} % for better arrays (eg matrices) in maths
%\usepackage{paralist} % very flexible & customisable lists (eg. enumerate/itemize, etc.)
\usepackage{verbatim} % adds environment for commenting out blocks of text & for better verbatim
%\usepackage{subfig} % make it possible to include more than one captioned figure/table in a single float
% These packages are all incorporated in the memoir class to one degree or another...

%%% HEADERS & FOOTERS
%\usepackage{fancyhdr} % This should be set AFTER setting up the page geometry
%\pagestyle{fancy} % options: empty , plain , fancy
%\renewcommand{\headrulewidth}{0pt} % customise the layout...
%\lhead{}\chead{}\rhead{}
%\lfoot{}\cfoot{\thepage}\rfoot{}

%%% SECTION TITLE APPEARANCE
%\usepackage{sectsty}
%\allsectionsfont{\sffamily\mdseries\upshape} % (See the fntguide.pdf for font help)
% (This matches ConTeXt defaults)

%%% ToC (table of contents) APPEARANCE
%\usepackage[nottoc,notlof,notlot]{tocbibind} % Put the bibliography in the ToC
%\usepackage[titles,subfigure]{tocloft} % Alter the style of the Table of Contents
%\renewcommand{\cftsecfont}{\rmfamily\mdseries\upshape}
%\renewcommand{\cftsecpagefont}{\rmfamily\mdseries\upshape} % No bold!

%%% END Article customizations

\title{HEP SW Collaboration: a few ideas...}
\author{Michel Jouvin, S\'ebastien Binet, David Rousseau}
%\date{} % Activate to display a given date or no date (if empty),
         % otherwise the current date is printed 

\begin{document}
\maketitle

\section{Context}

Next runs of LHC experiments and new generation of HEP experiments are
challenging HEP software with an unprecedented data deluge. At the
same time the budget constraint everywhere gives no choice but
impoving HEP software performance by a factor of magnitude in the next
5 to 10 years. HEP is not unique in facing such a challenge but has a
handicap: most of its computing problems are sequential in essence
when most of the performance improvement in new processor
architectures comes from parallelism (many cores, vector
instructions, etc\ldots).

HEP benefits from a rich but fragmented software ecosystem made of
many different types of packages covering simulation, analysis,
frameworks... Some of them are produced and maintain by large
collaborations (GEANT4, ROOT), others are developed and maintained by
experiments, sometimes in common like GAUDI, and many packages have
been started by an individual or a very small team. This is both a
strengh and a weakness for adressing the challenges ahead of us. This
is a strengh because we have a lot of people involved in SW
development, covering a wide range of expertises, and this diversity is
fostering innovation. On the other hand, this is also a weakness
because of the risk of effort duplication at a time where manpower is
limited if not in shortage. Some sort of coordination between projects
is the only possible answer to get the benefit of our diversity
without paying the price of the fragmentation.

HEP has a rich tradition of software development, assessed by its two
flagship projects, ROOT and GEANT4, now used outside the community. At
the same time, we have to learn from this history that almost all the
major software products now in use in the community have been started
by some individuals or groups to fullfil user/experiment needs but
never by a management decision. Sometimes, projects have been started
as "innovation" despite the management "hostility". At the same time,
those projects, in their fight to be recognized, were not always open
to new ideas and took time to recognize them. With this history, HEP
is not unique. The open-source model that has been so successful in
the last 10 years was invented has an answer to the same problems:
every project needs innovation and new ideas to evolve but a top-down
managed project has difficulty to integrate new contributors and
hardly benefits from new ideas. In this sense, the proposed HEP SW
collaboration is a way of recognizing that HEP learnt from the
open-source experience and its success to foster innovation and to get
many different parties involved in the same project.

The two main challenges we are facing are an efficient access to large
volume of distributed data and parallelization. A lot of expertise in
these areas exists outside HEP. Even though computing models may be
different, we can benefit from these expertises if we are able to
liaise with these other communities that include commercial actors,
in particular for Big Data. Several existing collaborations at the
local level also shown that the computer sciences are interested to
work with us as we are both a demanding use case and a community with
already a significant expertise allowing real collaborations.


\section{User stories}
To have a better understanding of how the HEP software collaboration could work, we propose below a few user stories. It is not meant to be complete.
%hosted projects %endorsed project

\subsection{"hosted" project }
I'm developing, alone or with a few others, a piece of software which I believe could be interesting to other groups. I apply to the HEP software collaboration 
with little more than one paragraph describing the project. The HEP software collaboration gives me its green light : at this point, the only criterion is that 
the project is somewhat relevant to HEP computing needs and challenges. From that point my project becomes a project {\em "hosted"}~\footnote{This word might not be the best choice.} 
by the HEP software collaboration. I have access to and benefits from a number of services provided by the HEP Software Collaboration. A non exclusive list could be:
\begin{itemize}
\item a software forge à la GitHub or Bitbucket providing both source code repositories, an issue tracker and a blog
\item a continuous integration infrastructure (nightly build and test infrastructure in particular)
\item other support for collaborative development, like mailing lists
\item documented choice of recommended best practices. I don't have to abide to these best practices but I know that they can help the adoption of my project by other groups and experiments. For example:
\begin{itemize}
\item choice of open source licenses
\item choice of coding rules and QA in general
\item development model
\item interfaces
\item packaging practices
\end{itemize}

\item limited access to test machines (real or virtual) with different architectures
\end{itemize}

For me, the real incentive to participate to the HEP SW collaboration was the software development infrastructure
that I can use for free and the visibility gained by my project. I do not expect any direct support or funding. I hope that
if my project is successful in raising interest for others, it will at least help to build a more sustainable community around
it or may be apply to some funding program for this kind of project. I appreciate that I'm the main person responsible for project
roadmap and strategy and that I am not at risk of seeing my project killed at some point by some decisions of the HEP SW Collaboration
management. I accept that if my project is really successful and becomes a component that several HEP packages rely on, I may be
asked to become an "endorsed project" with a greater control exerced by the collaboration.

The threshold to become a "hosted" project is very low to promote the creativity. 


\subsection{"endorsed" project }

I am or we are running a project that has become a corner stone of many other packages or HEP user activities or a specialized package of
interest for some of the computing challenges faced by HEP communities. I have already a robust software process in place and a governance model
for my project. I am ready to be more engaged with the HEP community but I want my project to continue its own life and have the ability to
develop relationships with other communities, even with different/competing needs. HEP Software Collaboration infrastructure is not the main incentive
for me but I recognize that what is offered is fulfilling my needs and may save some resources in my project spent to maintain our own infrastructure.
I appreciate that the HEP SW Collaboration allows me to use its infrastructure even though I am not a pure HEP project.

For applying to be labeled as an {\em "endorsed"}\footnote{This word might not be the best choice.} project, I understand that there is a formal
process, conversely to "hosted" projects. This requires to demonstrate to the HEP Software Collaboration Board why my project is of interest to the
HEP community, addressing in particular the following topics:
\begin{itemize}
\item relevance
\item performances 
\item compliance with the best practicies defined by the collaboration 
\item support (at least short and mid term) (e.g. the main developer is not at the end of her PhD) 
\end{itemize}

I understand that becoming an endorsed project may need some adaptation in my software process to comply with the collaboration best practies and will require
to integrate my documentation and user support in the collaboration framework. Being an "endorsed" project by the HEP Software Collaboration, the collaboration
will be invited to participate to my project governance, without taking a full control of it. The collaboration will act as some sort of "sponsor". Even though I'm aware
that the collaboration will not be a direct source of funding, it will give my project more funding opportunities, either by applying to some funding programs or 
by discussing with the funding
agencies relevant to me to help getting the adequate level of support. In return, I expect a much increased visibility, with the positive impact
on my project sustainability and my ability to attract new contributors.

Becoming an "endorsed" project may become a nature evolution of some "hosted" project but should not be the goal assigned to "hosted" projects. And
it should not be a requirement to be a "hosted project" before becoming an "endorsed" project, even though it will probably the case for many projects
started as R\&D activites: they will becomoe an "endorsed" project when reaching maturity. In particular in the initial phase of the collaboration,
it is envisionned that several existing projects have vocation to become immediately an "endorsed" projects.

\section{Goals}

Based on the previous user stories, we propose to define the following main goal for the HEP Software
Collaboration:

\begin{itemize}
\item
An umbrella organisation offering a lightweight coordination between projects, promoting collaboration between them
with the objective of improving software quality and visibility and ofreducing duplication when it is not motivated 
by innovation. The HEP Software Collaboration will in particular focus on fostering innovations that will help
to meet the two challenges mentioned in the introduction: the big data challenge and the ability to use efficiently
the new processor architectures.

\item
Set up a framework that will act as an incubator for new projects, offering them the necessary infrastructure
to adopt from the beginning a robust and sustainable development model, in line with HEP standards, and giving
them the visibility for the project to mature and expand beyond its original initiators.

\item
Be a point of contact between potential funding sources, in particular HEP experiments funding agencies, and the
projects endorsed or hosted by the collaboration. In particular, the HEP Software Collaboration will offer a
framework to facilitate consistent, complementary and non competing projects when applying to European funding
and to help with getting a coordinated funding from several sources from different countries/continents (e.g. European and US sources).

\item
Establish contacts with other scientific communities or parties that could be interested to contribute to HEP software
or to use it, widening the scope of certain projects. This seems particularly important to liaise with the Computer Science and Data Science
community that already showed interest for our challenges. Leveraging contacts existing at several institutes, the HEP
Software Collaboration could allow for a more formal and wider collaboration.

\end{itemize}

There is probably a lot to learn from the successful large software foundations that emerged in the last 10 years,
like the Apache or Eclipse foundations, even though we think that, at least in the beginning, direct funding of the
projects by the collaboration will be marginal if not unlikely.

\section{Development Model and Tools}

Software development models evolved dramatically in the last decade as
a result of two different processes:

\begin{itemize}
\item 
Emergence of Agile methodologies: breaking from traditional
waterfall metodologies where the iteration cycle is very slow, agile
methodologies put user needs ("user stories") at the center of the
development process with short iteration cycles and demonstration
at the end of each development cycle.
The result is a user-driven evolution of the product, one of the
characteristic of the most successful software packages both in HEP
and outside the community. 
The HEP software inventory recently made by P. Elmer pointed out that
this was an important feature shared by all successful tools and
packages in our community.
ROOT has been an early adopter of this methodology and demonstrated
it could be successful at a large scale.

\item 
Social coding as implemented by successful platforms like GitHub and
BitBucket.
These platforms allow an easy aggregation of external or occasional
contributors and provide tools helping the communication between
project members, making the management of a project reasonnably
easy even with a large number of contributors.

\end{itemize}

Based on these recognized evolution and on our current practices, the HEP Software Collaboration should define
best practices regarding development model (e.g. public code access, importance of unit testing), documentation, user support\ldots

To help projects benefit from these successful practices without wasting resources to
operating/maintaining duplicated software development infrastructures, the HEP Software Collaboration will
set up and operate an infrastructure open to every project part of the collaboration, either as a
"hosted" or as an "endorsed" project. Using every component of this infrastructure should not be mandatory,
even though the HEP Software Collaboration could think about some incentive (gamification?) to use it.  

\begin{itemize}

\item mailing list both for intra-project communication and for public communication (release announcements...)

\item hosting of source code repositories with the largest sustainable choice of VCS. Being prescriptive for the VCS
to use is generally a source of resistance... On the other hand, we could restrict the hosting to the most popular DVCS
(Git and Mercurial in particular), as probably centralized VCS like SVN don't really allow to implement seamlessly the
proposed models.

\item continuous integration with the most popular tools (e.g. Drone, Travis, Jenkins). Again, the collaboration will
have to find the right balance between the available manpower to setup and maintain the infrastructure and the diversity
required to be attractive for projects without being too prescriptive.

\item an infrastructure to build nightlies with the appropriate dashboards to easily identify problematic components,
source of errors...

\item a service to host project documentation in a wiki-like (lightweight) format. Ideally, this
service should make easy collective contribution to the documentation with a peer-review (lightweight) process. A good example
is GitHub Pages service based on Jekyll, based on the contents of a Git repository where contributions can be done through pull
requests.

\end{itemize}

All this infrastructure will be set to foster inter-project compatibilities and enabled an efficient cross-pollination of
the projects in the HEP Software Collaboration. The collaboration may want to define, as proposed best practices, the different
approaches possible for project interoperabilities (see P. Mato's talk at the kickoff meeting), with a particular focus on
data compatibility.


\section{IPR}

At least in the initial stage of the HEP Software Collaboration, the collaboration should not be too prescriptive about IPR policy.
The only real requirement with "hosted" or "endorsed" project should be that they use an open-source license compatible with a
copyleft model à la BSD or Apache2, as a too restrictive open-source license could prevent reusing the package in others.

The HEP Software Collaboration may discuss the opportunity to create its own open-source license, derived from one of well-known one,
and propose it as the license for the collaboration projects, without being prescriptive again. It is expected that several projects
"hosted" or "endorsed" by the collaboration will have to take into account constraints from other parties and the collaboration should
make it possible rather than difficult/impossible... 

Also, for some flagship projects or when request by projects, the HEP Software Collaboration could take ownership of the IPRs. This
is probably a difficult topic, with potentially conflicting interests and legal difficulties, and this should not be considered as a 
prerequisite. This could be a topic discussed by HEP Software Collaboration governance after it has been established.


\section{Governance}

The HEP Software Collaboration shoud start with a governance as lightweight as possible. We should avoid at all price to build
a bureaucracy or to give the impression that the collaboration governance will take control of the projects. This governance should
really help to implement this bottom-up, agile approach to software development and make it clear that there is no attempt to
build a prescriptive governance that will impose choices and kill innovation at a time we need it.

The collaboration governance model should be inspired by the existing software foundation, like the Apache foundation, where projects
retain a strong personality and their own technical/political governance. If we want to foster the collaboration spirit between projects,
there is probably no choice but decisions by consensus (or quasi-consensus) on important matters.

The collaboration should be agile in building its governance! We should not try to setup the definitive, almost-perfect, one! But rather
be as minimalistic as we can at the beginning and evolve/refine it, based on the experience. Our proposal is to start with 2 boards:

\begin{itemize}

\item Technical Board: this should be the main board of the collaboration. It should be seen as a forum between the projects that are part
of the collaboration and be open to both "endorsed" and "hosted" projects. Its main focus will be discussion and strategic/technical choices for 
the development infrastructure run by the collaboration and the identification of potential commonalities between projects. Its main objective should
be to reach consensus in decisions but in controversial circumstances, if neeeded, the "endorsed" project consensus will prevail. Any really controversial
issue that cannot benefit from further discussions before decisions will have to be taken or at least endorsed by the Scientific Board.

\item Scientific Board: it will have the responsability to define the long term strategy, the funding implications and discuss evolution of the governance if
needed. It should include representatives of the main institues contributing to the collaborations, including small/medium size ones, representatives of the main
user communities (e.g. HEP experiments) and a few representatives from the Technical Board. Consensus should be the rule for decisions.

\end{itemize}

One of the most important challenge for the HEP Software Collaboration in this first stage is to be really HEP-wide and not
to appear as an appendix of CERN or even the only LHC experiments. We should really ensure that the members these
boards reflect this objective to embrace the whole HEP and possibly in the future to extend to other communities interested.
We should also ensure that small/medium size institutes participate to this governance.


\section{Funding}

In its initial stage, the HEP Software Collaboration should not aim to directly funding the projects. Projects funding will come
either from the existing source for "hosted" projects or from applying to different national or continental funding programs. In
particular for European funding, the HEP Software Collaboration will ensure that there is no destructive competition for funding
between the different collaboration projects and will help to get the right consortium set up for building proposals and to emphasize 
the community interest in the project.

Parties involved in projects or in the computing activities of HEP communities will be encourage to apply for complementary national
funding when there are opportunities. This may not be necessarily for one particular software project in the collaboration but could be
the occasion of a "mini-collaboration" at the national level, covering several software projects from the HEP Software Collaboration. In this
particular case, the HEP Software Collaboration will ensure the appropriate coordination between the HEP-wide activities and the national ones.


\end{document}



