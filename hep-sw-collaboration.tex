% !TEX TS-program = pdflatex
% !TEX encoding = UTF-8 Unicode

\documentclass[11pt]{article} % use larger type; default would be 10pt

\usepackage[utf8]{inputenc} % set input encoding (not needed with XeLaTeX)

%%% Examples of Article customizations
% These packages are optional, depending whether you want the features they provide.
% See the LaTeX Companion or other references for full information.

%%% PAGE DIMENSIONS
\usepackage{geometry} % to change the page dimensions
\geometry{a4paper} % or letterpaper (US) or a5paper or....
% \geometry{margin=2in} % for example, change the margins to 2 inches all round
% \geometry{landscape} % set up the page for landscape
%   read geometry.pdf for detailed page layout information

\usepackage{graphicx} % support the \includegraphics command and options

% \usepackage[parfill]{parskip} % Activate to begin paragraphs with an empty line rather than an indent

%%% PACKAGES
%\usepackage{booktabs} % for much better looking tables
\usepackage{array} % for better arrays (eg matrices) in maths
%\usepackage{paralist} % very flexible & customisable lists (eg. enumerate/itemize, etc.)
\usepackage{verbatim} % adds environment for commenting out blocks of text & for better verbatim
%\usepackage{subfig} % make it possible to include more than one captioned figure/table in a single float
% These packages are all incorporated in the memoir class to one degree or another...

%%% HEADERS & FOOTERS
%\usepackage{fancyhdr} % This should be set AFTER setting up the page geometry
%\pagestyle{fancy} % options: empty , plain , fancy
%\renewcommand{\headrulewidth}{0pt} % customise the layout...
%\lhead{}\chead{}\rhead{}
%\lfoot{}\cfoot{\thepage}\rfoot{}

%%% SECTION TITLE APPEARANCE
%\usepackage{sectsty}
%\allsectionsfont{\sffamily\mdseries\upshape} % (See the fntguide.pdf for font help)
% (This matches ConTeXt defaults)

%%% ToC (table of contents) APPEARANCE
%\usepackage[nottoc,notlof,notlot]{tocbibind} % Put the bibliography in the ToC
%\usepackage[titles,subfigure]{tocloft} % Alter the style of the Table of Contents
%\renewcommand{\cftsecfont}{\rmfamily\mdseries\upshape}
%\renewcommand{\cftsecpagefont}{\rmfamily\mdseries\upshape} % No bold!

%%% END Article customizations

\title{HEP SW Collaboration: a few ideas...}
\author{Michel Jouvin, S\'ebastien Binet, David Rousseau}
%\date{} % Activate to display a given date or no date (if empty),
         % otherwise the current date is printed 

\begin{document}
\maketitle

\section{Context}

Next runs of LHC experiments and new generation of HEP experiments are
challenging HEP software with an unprecedented data deluge. At the
same time the budget constraint everywhere gives no choice but
impoving HEP software performance by a factor of magnitude in the next
5 to 10 years. HEP is not unique in facing such a challenge but has a
handicap: most of its computing problems are sequential in essence
when most of the performance improvement in new processor
architectures comes from parallelism (many cores, vector
instructions, etc\ldots).

HEP benefits from a rich but fragmented software ecosystem made of
many different types of packages covering simulation, analysis,
frameworks... Some of them are produced and maintain by large
collaborations (GEANT4, ROOT), others are developed and maintained by
experiments, sometimes in common like GAUDI, and many packages have
been started by an individual or a very small team. This is both a
strengh and a weakness for adressing the challenges ahead of us. This
is a strengh because we have a lot of people involved in SW
development, covering a wide range of areas, and this diversity is
fostering innovation. On the other hand, this is also a weakness
because of the risk of efforts duplication at a time where manpower is
limited if not in shortage. Some sort of coordination between projects
is the only possible answer to get the benefit of our diversity
without paying the price of the fragmentation.

HEP has a rich tradition of software development, assessed by its two
flagship projects, ROOT and GEANT4, now used outside the community. At
the same time, we have to learn from this history that almost all the
major software products now in use in the community have been started
by some individuals or groups to fullfil user/experiment needs but
never by a management decision. Sometimes, projects have been started
as "innovation" despite the management "hostility". At the same time,
those projects, in their fight to be recognized, were not always open
to new ideas and took time to recognize them. With this history, HEP
is not unique. The open-source model that has been so successful in
the last 10 years was invented has an answer to the same problems:
every project needs innovation and new ideas to evolve and a top-down
managed project has difficulty to integrate new contributors and
hardly benefits from new ideas. In this sense, the proposed HEP SW
collaboration is a way of recognizing that HEP learnt from the
open-source experience and its success to foster innovation and to get
many different parties involved in the same project.

The two main challenges we are facing are an efficient access to large
volume of distributed data and parallelization. A lot of expertise in
these areas exists outside HEP. Even though computing models may be
different, we can benefit from these expertises if we are able to
liaise with these other communities that include commercial actors,
in particular for Big Data. Several existing collaborations at the
local level also shown that the computer sciences is interested to
work with us as we both a demanding use case and a community with
already a significant expertise allowing real collaborations.

\section{Goals}

\begin{itemize}
\item
Umbrella organisation offering a lightweight coordination,

\item
Optimize the effort on different kinds of software by
avoiding/limiting duplications when not source of innovation,

\item
Foster innovation through incubator activities,

\item
Learn from the major software foundation experiences: Apache, Eclipse.

\end{itemize}

\section{Development Model}

Software development models evolved dramatically in the last decade as
a result of two different processes:

\begin{itemize}
\item 
Emergence of Agile methodologies: breaking from traditional
waterfall metodologies where the iteration cycle is very slow, agile
methodologies put user needs ("user stories") at the center of the
development process with short iteration cycles and demonstration
at the end of each development cycle.
The result is a user-driven evolution of the product, one of the
characteristic of the most successful software packages both in HEP
and outside the community. 
The HEP software inventory recently made by P. Elmer pointed out that
this was an important feature shared by all successful tools and
packages in our community.
ROOT has been an early adopter of this methodology and demonstrated
it could be successful at a large scale.

\item 
Social coding as implemented by successful platforms like GitHub and
BitBucket.
These platforms allow an easy aggregation of external or occasional
contributors and provide tools helping the communication between
project members, making the management of a project reasonnably
easy even with a large number of contributors.

%% SB: mention the need for: 
%%   - mailing-list|group-list for communication (intra-project,
%%   release-annoucements, ...)
%%
%%   - continuous integration (drone.io, travis-ci.org,
%%   jenkins-ci.org)
%%
%%   - infrastructure for code coverage, nightlies (?),... QA.
%%   none should be mandatory, but a system of gamification (with
%%   badges) would help to increase the incentive to improve QA.

%% SB: we should mention inter-project compatibilites.
%%     -> this is paramount to get a good n-projects cross-pollination
%%     -> mention the 3-layers from P. Mato talk
%%     -> requiring a C-API for every project is probably too heavy
%%     handed
%%     -> require simple C-compatible binary layouts for data would
%%     probably suffice in the majority of cases.

\end{itemize}


\section{IPR}

Take ownership of software IPR rights?

Promote a unique license based on successful open-source licenses like
Apache2 ?

%% License: why not BSD ? (Apache2 is IIRC BSD-2 + lawyer-ing)

\section{Governance}

AGILE !

\section{Funding}

???

Role of national/local projects

\end{document}



