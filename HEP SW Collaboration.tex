% !TEX TS-program = pdflatex
% !TEX encoding = UTF-8 Unicode

\documentclass[11pt]{article} % use larger type; default would be 10pt

\usepackage[utf8]{inputenc} % set input encoding (not needed with XeLaTeX)

%%% Examples of Article customizations
% These packages are optional, depending whether you want the features they provide.
% See the LaTeX Companion or other references for full information.

%%% PAGE DIMENSIONS
\usepackage{geometry} % to change the page dimensions
\geometry{a4paper} % or letterpaper (US) or a5paper or....
% \geometry{margin=2in} % for example, change the margins to 2 inches all round
% \geometry{landscape} % set up the page for landscape
%   read geometry.pdf for detailed page layout information

\usepackage{graphicx} % support the \includegraphics command and options

% \usepackage[parfill]{parskip} % Activate to begin paragraphs with an empty line rather than an indent

%%% PACKAGES
%\usepackage{booktabs} % for much better looking tables
\usepackage{array} % for better arrays (eg matrices) in maths
%\usepackage{paralist} % very flexible & customisable lists (eg. enumerate/itemize, etc.)
\usepackage{verbatim} % adds environment for commenting out blocks of text & for better verbatim
%\usepackage{subfig} % make it possible to include more than one captioned figure/table in a single float
% These packages are all incorporated in the memoir class to one degree or another...

%%% HEADERS & FOOTERS
%\usepackage{fancyhdr} % This should be set AFTER setting up the page geometry
%\pagestyle{fancy} % options: empty , plain , fancy
%\renewcommand{\headrulewidth}{0pt} % customise the layout...
%\lhead{}\chead{}\rhead{}
%\lfoot{}\cfoot{\thepage}\rfoot{}

%%% SECTION TITLE APPEARANCE
%\usepackage{sectsty}
%\allsectionsfont{\sffamily\mdseries\upshape} % (See the fntguide.pdf for font help)
% (This matches ConTeXt defaults)

%%% ToC (table of contents) APPEARANCE
%\usepackage[nottoc,notlof,notlot]{tocbibind} % Put the bibliography in the ToC
%\usepackage[titles,subfigure]{tocloft} % Alter the style of the Table of Contents
%\renewcommand{\cftsecfont}{\rmfamily\mdseries\upshape}
%\renewcommand{\cftsecpagefont}{\rmfamily\mdseries\upshape} % No bold!

%%% END Article customizations

\title{HEP SW Collaboration: a few ideas...}
\author{Michel Jouvin, Sébastien Binet}
%\date{} % Activate to display a given date or no date (if empty),
         % otherwise the current date is printed 

\begin{document}
\maketitle

\section{Context}

Next runs of LHC experiments and new generation of HEP experiments are challenging HEP software with an unprecedented data deluge. At the same time the budget constraint everywhere gives no choice but impoving HEP software performance by a factor of magnitude in the next 5 to 10 years. HEP is not unique in facing such a challenge but has a handicap: most of its computing problems are sequential in essence when most of the performance improvement in new processor architectures comes from parallelism (many cores, vector instructions....).

HEP benefits from a rich but fragmented software ecosystem made of a many different type of packages covering simulation, analysis, frameworks... Some of them are produced and maintain by large collaborations (GEANT4, ROOT), others are developed and maintained by experiments, sometimes in common like GAUDI, and many packages have been started by an individual or a very small team. This is both a strengh and a weakness for adressing the challenges ahead of us. This is a strengh because we have a lot of people involved in SW development, covering a wide range of areas, and this diversity is fostering innovation. On the other hand, this is also a weakness because of the risk of effort duplication at a time where the manpower is limited if not in shortage. Some sort of coordination between projects is the only possible answer to get the benefit of our diversity without paying the price of the fragmentation.

HEP has a rich tradition of software development, assessed by its two flagship projects, ROOT and GEANT4, now used outside the community. At the same time, we have to learn from this history that almost all the major software products now in use in the community have been started by some individuals or groups to fullfil user/experiment needs but never by a management decision. Sometimes, projects have been started as "innovation" despite the management "hostility". At the same time, those projects, in their fight to be recognized, were not always open to new ideas and took time to recognize them. With this history, HEP is not unique. The open-source model that has been so successful in the last 10 years was invented has an answer to the same problems: every project needs innovation and new ideas to evolve and a top-down managed project has difficulty to integrate new contributors and hardly benefits from new ideas. In this sense, the proposed HEP SW collaboration is a way of recognizing that HEP learnt from the open-source experience and its success to foster innovation and to get many different parties involved in the same project.

\section{Goals}

\section{Development Model}

\section{IPR}

\section{Governance}


\end{document}



